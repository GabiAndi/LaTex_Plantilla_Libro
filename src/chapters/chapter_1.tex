\chapter{Tipos de documentos de \LaTeX}

\pagestyle{empty}

En este capítulo vamos a ver los conceptos básicos de \LaTeX. Crearemos documentos y aprenderemos a utilizar la plantilla, con lo cual usted podrá observar ejemplos de los distintos estilos contenidos.

\newpage
\pagestyle{fancy}

\section{Ejemplo de sección}

Este capítulo es para mostrarle como es la plantilla. Usted puede citar \cite{Rojas2018}.

\subsection{Ejemplo de subsección}

Puede hacer cajas de colores:

\begin{info}{Título de la caja}
    Añadir contenido para resaltar su importancia
\end{info}

\subsubsection{Subsubsección}

También pude definir tipos de cajas, como este teorema:

\begin{theo}{Sumatoria de números}{id_para_reconocer}
    Para todo $n$ natural:

    \begin{equation}
        \sum\limits_{i=1}^n i = \frac{n(n+1)}{2}
    \end{equation}
\end{theo}

Luego podemos hacer referencia a \ref{teorema:id_para_reconocer} para poder explicar su contenido.

\subsubsection{Figuras y tablas}

\begin{figure}[H]
    \centering
    \includesvg[inkscapelatex=false,width=0.3\textwidth]{src/images/doc/logo}
    \caption{Ejemplo de figura}
    \label{fgr:logo}
\end{figure}

\begin{table}[H]
    \begin{center}
        \begin{tabular}{|l|l|}
            \hline
            \multicolumn{2}{|c|}{Sopladores de aireación} \\ \hline
            Tipo & Desplazamiento positivo \\
            Marca & Repicky \\
            Modelo & R3.0 \\
            Capacidad & 1.631$m^{3} / h$ \\
            Tamaño del motor & 75HP \\ \hline
        \end{tabular}
    \end{center}

    \caption{Características de los Sopladores Repicky R3.0.}
    \label{fgr:repicky}
\end{table}

\section{Lista de tareas}

Podemos realizar un listado con tareas y demás:

\begin{todolist}
    \item{Abstract.}
    \item[\done]{Descripción del proyecto:}
    \begin{todolist}
        \item[\done]{Introducción a la problematica.}
        \item[\done]{Análisis de solución.}
        \item[\done]{Planificación del proyecto.}
    \end{todolist} 
    \item[\done]{Desarrollo técnico:}
    \begin{todolist}
        \item[\done]{Selección de componentes.}
        \item[\done]{Diseño del sistema.}
        \item[\done]{Cálculo de presupuesto.}
        \item[\done]{Pruebas en prototipo.} 
    \end{todolist} 
    \item{Conclusiones.}
    \item[\done]{Anexos:}
    \begin{todolist}
        \item[\done]{Esquemas y planos.}
        \item[\done]{Desarrollos complementarios.}
    \end{todolist} 
\end{todolist}

\section{Sección con contenido largo}

\lipsum[1-10]

\newpage

\section{Página horizontal}